\documentclass{article}
\usepackage[letterpaper, margin=1in]{geometry}
\usepackage{amsmath}
\usepackage{amssymb}
\usepackage[T1]{fontenc}
\usepackage{fancyhdr}
\usepackage{lastpage}
\usepackage{graphicx}
\usepackage{fontawesome}
\usepackage{setspace}
\usepackage{hyperref}

\pagestyle{fancy}
\fancyhf{}
\renewcommand{\headrulewidth}{0pt}
\setlength{\hoffset}{-2pt}
\setlength{\footskip}{20pt}
\cfoot{Last updated: August 11, 2024}
\rfoot{Page \thepage\hspace*{3pt}$\vert$\hspace*{3pt}\pageref{LastPage}}

\newcommand\blfootnote[1]{%
  \begingroup
  \renewcommand\thefootnote{}\footnote{#1}%
  \addtocounter{footnote}{-1}%
  \endgroup
}

\begin{document}

\small
\vspace*{-44pt}

\begin{center}
    {\LARGE\scshape Javed Iqbal}\\
    \vspace{3pt}
    \faPhone{} +92 3064759479 \quad \faEnvelope{} \href{mailto:iqbaljaved1921@gmail.com}{iqbaljaved1921@gmail.com} \quad \faLinkedin{} \href{https://www.linkedin.com/in/jiqbal1921}{LinkedIn} \quad \faGithub{} \href{https://github.com/Jiqbal1921}{GitHub}
\end{center}

\section*{Education}
\vspace{-\baselineskip}
\noindent\makebox[\linewidth]{\rule{\textwidth}{0.4pt}}
\textbf{Master of Science in Astronomy and Astrophysics} \hfill{GPA 3.24/4.0} \\
\href{https://www.ist.edu.pk/}{Institute of Space Technology}, Islamabad \hfill \textit{September 2019 -- August 2023}\\
\textbf{Thesis Title}\\
\textit{Photometric search for short-period Jupiter-size exoplanets in open clusters NGC884 and NGC869} \\
\textbf{Supervisors}\\
\href{https://faculty.tarleton.edu/goderya/}{Dr. Shaukat Goderya (Professor)}, 
\href{https://www.ist.edu.pk/fazeel-mahmood-khan}{Dr. Fazeel Mahmood Khan (Professor)} 
\\ \\
\textbf{Bachelor of Science in Physics} \hfill{GPA 3.60/4.0} \\
\href{https://bzu.edu.pk/}{Bahauddin Zakariya University}, Multan \hfill \textit{August 2015 -- April 2019}\\
\textbf{Thesis Title}\\
\textit{Study of electrical properties of single-wall carbon nanotube in field-effect transistor}\\
\textbf{Supervisor}\\
\href{https://www.uosahiwal.edu.pk/depart-hod/physics}{Dr. Hafiz Tariq Masood (Assistant Professor)}

\section*{Research Interests}
\vspace{-\baselineskip}
\noindent\makebox[\linewidth]{\rule{\textwidth}{0.4pt}}
I am interested in discovering and characterizing exoplanets and brown dwarfs using the Transit Photometric method with ground-based and space-based telescopes. I am also interested in studying exoplanet atmospheres, demographics, and formation.

\section*{Research Experience}
\vspace{-\baselineskip}
\noindent\makebox[\linewidth]{\rule{\textwidth}{0.4pt}}
 \textbf{Master’s Thesis} \\
    \textit{Institute of Space Technology \hfill Islamabad, Pakistan} \\
    \textbf{Photometric search for short-period Jupiter-size exoplanets in open clusters NGC884 and NGC869} \\
    \textbf{Summary:} The target open clusters NGC 884 and NGC 869 were observed remotely for 11 nights using the 0.8-meter telescope at Tarleton State University (TSU). Observations were conducted in the VBRI bandpass for a total of 27.9 hours, and light curves were extracted using a difference image subtraction package (ISIS 2.0) and a variability search toolkit (VaST). The Box Least Squares Algorithm (BLS) was applied to search for planetary transit events, but no such events were detected in the dataset. Instead, a total of 25 variable stars were identified, including four promising new candidates. The periods of all these variable stars were calculated using the Lomb-Scargle periodogram and the Lafler-Kinman method, and classification of variability was made based on their periods, shape, and amplitude of the light curves. \\ \\
\textbf{Graduate Research Assistant} \\
    \textit{Astronomy and Astrophysics Research Lab, IST \hfill Islamabad, Pakistan} \\
    \textbf{ Project: Capabilities and Limitations of IST Observatory} \\
    \textbf{Summary:} The goal of the project was to assess the capabilities of the IST observatory, focusing on observational limits and equipment efficiency under the challenging weather conditions of Islamabad, tested through photometric observations of variable stars and exoplanets, and identify potential improvements to the facility.

     
\section*{Publications}
\vspace{-\baselineskip}
\noindent\makebox[\linewidth]{\rule{\textwidth}{0.4pt}}

\begin{itemize}
    \item \textbf{J. Iqbal}, \textit{Photometric Search for Short-Period Jupiter-Size Exoplanets in Open Clusters NGC 884 and NGC 869,} under review in \textbf{OEJV}.
    \item \textbf{J. Iqbal}, \textit{New Transit Analysis of WASP-10b: A study with Exoplanet Watch / PACS E Lab,} under review in \textbf{A\&A}.
    \end{itemize}
\section*{Presentations}
\vspace{-\baselineskip}
\noindent\makebox[\linewidth]{\rule{\textwidth}{0.4pt}}
\textbf{Poster Presentations}
\begin{itemize}
    \item \textit{Photometry and Transit Modeling of the Exoplanet TOI-2109b} (POP Poster), Sagan Summer Workshop, NASA Exoplanet Science Institute, July 22, 2024. \href{https://nexsci.caltech.edu/workshop/2024/posters/Poster_JavedIqbal_49.pdf}{(Link)}
    \item \textit{Determining the Parameters of Open Clusters NGC 884 and NGC 869 Using Gaia EDR3,} International Conference on Relativistic Astrophysics and Cosmology (ICRAC-24), COMSATS University, Lahore,  February 02, 2024. \href{https://lahore.comsats.edu.pk/ICRAC2024/program.aspx}{(Link)}
\end{itemize}

\textbf{Oral Talks}
\begin{itemize}
    \item \textit{Search for Short-Period Jupiter-Size Exoplanets in Open Clusters NGC 884 and NGC 869 using Transit Photometry,} World Space Week Seminar, National University of Sciences and Technology (NUST), Islamabad, October 9, 2023.
    \item \textit{Exoplanet Detection Using Transit Photometry,} Master Research Seminar, Institute of Space Technology, Islamabad, November 25, 2022.
    \end{itemize}
    
\section*{Conferences, Workshops and Seminars}
\vspace{-\baselineskip}
\noindent\makebox[\linewidth]{\rule{\textwidth}{0.4pt}}
\begin{itemize}
    \item 2024 Sagan Exoplanet Summer Hybrid Workshop on \textit{Advances in Direct Imaging: From Young Jupiters to Habitable Earths} (July 24-28)(Virtual).
    \item 2023 Sagan Exoplanet Summer Hybrid Workshop on \textit{Characterizing Exoplanet Atmospheres: The Next Twenty Years} (July 24-28)(Virtual).
    \item Europlanet Summer School on \textit{Ground-based observations and science communication} (8-18 August 2023)(Virtual).
    \item Workshop on \textit{General Relativity, Cosmology, and Astrophysics} on November 2023 organized by the Institute of Space Technology, Islamabad.
    \item Webinar on \textit{Two-fluid plasma model for generation of seed magnetic fields and astrophysical jets} Organized by COMSTECH, Islamabad.
    \item Fourth Abdus Salam Memorial Lecture on \textit{The Time of Your Life} by Dr. Robert L. Jaffe (2018) at Lahore University of Management Sciences (LUMS), Lahore, Pakistan.
\end{itemize}

\section*{Teaching and Mentoring Experience}
\vspace{-\baselineskip}
\noindent\makebox[\linewidth]{\rule{\textwidth}{0.4pt}}
\begin{itemize}
    \item \textbf{Physics Lecturer,} STEM Institute PWD Campus, Islamabad \hfill \textit{September 2021 -- August 2022} \\
     Prepared and delivered lectures, developed course materials and assessments, and designed engaging class activities.
    \item \textbf{Graduate Mentor,} IST Astronomy Camp 2023 \hfill
    \textit{May 2023 -- July 2023} \\
    Mentored two undergraduate students on astronomy projects, providing guidance in telescope operation, observations, data analysis, and in preparing project reports and presentations.
\end{itemize}

\section*{Honors and Awards }
\vspace{-\baselineskip}
\noindent\makebox[\linewidth]{\rule{\textwidth}{0.4pt}}
\begin{itemize}
    \item Preliminary Asteroid Discovery, International Astronomical Search Collaboration \href{http://iasc.cosmosearch.org/}{(IASC)} \hfill 2022 
    \item Master Research Grant, Institute of Space Technology, Islamabad \hfill 2020 
    \item Academic Scholarship, Bahauddin Zakariya University \hfill 2018 
    \item Prime Minister's Youth Laptop Scheme \href{https://laptop.pmyp.gov.pk/index.php}{(PMYLS)} \hfill 2018 
    \item 3\textsuperscript{rd} Position, project presentation, Bahauddin Zakariya University \end{itemize}

\section*{Technical Skills}
\vspace{-\baselineskip}
\noindent\makebox[\linewidth]{\rule{\textwidth}{0.4pt}}
\begin{itemize}
    \item \textbf{Programming:} Python, Git, Bash
    \item \textbf{Academic Applications:} LaTeX, Microsoft Office
    \item \textbf{Telescope Operation:} Extensive experience with 16-inch Meade Telescope and 0.8-meter robotic telescope. Over 100 hours of observational time
    \item \textbf{Observation Planning:} Expertise in designing and executing observational plans for variable star and exoplanet photometric data acquisition, with over 20 successful observation plans co
    \item \textbf{Data Reduction:}  IRAF, ISIS, VaST, FITSH, AstroImageJ, Photutils
    \item \textbf{Time Series Analysis:} PyTransit, VARTOOLS, Batman
\end{itemize}

\section*{Professional Affiliation}
\vspace{-\baselineskip}
\noindent\makebox[\linewidth]{\rule{\textwidth}{0.4pt}}
\begin{itemize}
    \item \textbf{American Association of Variable star Observers} 
 \href{https://www.aavso.org/users/jiqbal}{(AAVSO)}} \hfill 2023 - Present \\ Member
\end{itemize}

\section*{Leadership}
\vspace{-\baselineskip}
\noindent\makebox[\linewidth]{\rule{\textwidth}{0.4pt}}
\begin{itemize}
    \item \textbf{\href{https://skyversemagazine.org/team-member/}{SkyVerse Magazine Pakistan}} \hfill 2020 -- Present \\
    Founder and CEO
\end{itemize}

\section*{References}
\vspace{-\baselineskip}
\noindent\makebox[\linewidth]{\rule{\textwidth}{0.4pt}}
\begin{itemize}
    \item \textbf{Dr. Shaukat Goderya, Professor} \\
    Director Programs for Astronomy Education and Research \\ Tarleton State University (TSU), Texas, USA. \\
    \textbf{Email:} \href{mailto:goderya@tarleton.edu}{goderya@tarleton.edu}
    \item \textbf{Dr. Fazeel Mahmood Khan,  Professor}  \\
    Department of Space Sciences \\
    Institute of Space Technology (IST), Islamabad, Pakistan. \\
    \textbf{Email:} \href{mailto:fazeel@mail.ist.edu.pk}{fazeel@mail.ist.edu.pk}
\end{itemize}

\blfootnote{Created using \LaTeX}
\end{document}



